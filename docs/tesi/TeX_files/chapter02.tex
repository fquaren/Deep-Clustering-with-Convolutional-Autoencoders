\chapter{Introduction}

% This chapter contains all the informations on the concepts used in this thesis.
% sources: Deep learning, I. Goodfellow et. al

Introduction:
* machine learning -> deep learning

From Ch. 5 of Goodfellow:

"A machine learning algorithm is an algorithm that is able to learn from data."

Mitchell (1997) “A computer program is said to learn from experience E with respect to some class of tasks T and performance measure P , if its performance at tasks in T , as measured by P, improves with experience E.” 

"Machine learning allows us to tackle tasks that are too difficult to solve with fixed programs written and designed by human beings."

"Machine learning tasks are usually described in terms of how the machine learning system should process an example. An example is a collection of features that have been quantitatively measured from some object or event that we want the machine learning system to process. [...]  For example, the features of an image are usually the values of the pixels in the image.

"Classification: In this type of task, the computer program is asked to specify which of k categories some input belongs to."

"In order to evaluate the abilities of a machine learning algorithm, we must design a quantitative measure of its performance. Usually this performance measure P is specific to the task T being carried out by the system."

"For tasks such as classification, classification with missing inputs, and transcription, we often measure the accuracy of the model. Accuracy is just the proportion of examples for which the model produces the correct output."

"Usually we are interested in how well the machine learning algorithm performs on data that it has not seen before, since this determines how well it will work when deployed in the real world. We therefore evaluate these performance measures using a test set of data that is separate from the data used for training the machine learning system."

"Machine learning algorithms can be broadly categorized as unsupervised or supervised by what kind of experience they are allowed to have during the learning process. Most of the learning algorithms in this book can be understood as being allowed to experience an entire dataset. A dataset is a collection of many examples, as defined in section 5.1.1. Sometimes we will also call examples data points.

"Unsupervised learning algorithms experience a dataset containing many features, then learn useful properties of the structure of this dataset. In the context of deep learning, we usually want to learn the entire probability distribution that generated a dataset, whether explicitly as in density estimation or implicitly for tasks like synthesis or denoising. Some other unsupervised learning algorithms perform other roles, like clustering, which consists of dividing the dataset into clusters of similar examples."

"Supervised learning algorithms experience a dataset containing features, but each example is also associated with a label or target. "

***``Roughly speaking, unsupervised learning involves observing several examples of a random vector x, and attempting to implicitly or explicitly learn the proba- bility distribution p(x), or some interesting properties of that distribution, while supervised learning involves observing several examples of a random vector x and an associated value or vector y, and learning to predict y from x, usually by estimating p(y | x ).the target y being"

5.8.2
k-means Clustering
...

Chapter 9
Convolutional Networks (~40pages)

Chapter 14
Autoencoders


* clustering for classification 
?
* deep learning in medical sciences
* use cases for deep learning in medical sciences
* CT, MRI, PET -> focus on how images are different from one

